\documentclass{article}
\usepackage{fancyhdr}
\usepackage{graphicx}
\usepackage{amsmath}
\usepackage{xcolor}
\usepackage[margin=1in]{geometry}

\pagestyle{fancy}
\graphicspath{ {./img/} }

\begin{document}
	\begin{titlepage}
		\begin{center}
			\vspace{1cm}
			{\LARGE\textbf{Implementing and Testing Logisim Circuit using the Assembler and Debugger}}

			\vspace{1.5cm}
			\textbf{\large Ghassan Arnouk}\\
			
			\vspace{1cm}
			\large SYSC 3006A\\
			\large Summer 2020\\
			\large Lab 6 Report\\
			\large Group 1\\
			
						
			\vspace{2cm}
			\textbf{Instructor:} Michel Sayde\\
			
			\vspace{0.1cm}
			\textbf{TA:} Khalid Almahrog\\
			
			\vspace{0.1cm}
			\textbf{Submitted:} 2020/06/16\\			
		\end{center}
	\end{titlepage}
	
	\lhead{Ghassan Arnouk (Group 1)}
	\rhead{Implementing and Testing Logisim Circuit}
	\pagebreak
	
	\section{Fragment 1}
	\subsection{Complete the Fragment1SRC code by replacing all occurrences of “***” with the necessary details. Do not add any additional instructions. Then Submit your completed (working)	Fragment1SRC code.}
	Please find \textbf{\emph{Fragment1-SRC-myAnswers.txt}} file in \textbf{Lab 6} repository.
	
	\section{Fragment 2}
	\subsection{Complete the Fragment2SRC code by replacing all occurrences of “***” with the	necessary details. Do not add any additional instructions. Then Submit your completed (working)	Fragment2SRC code}
	Please find \textbf{\emph{Fragment1-SRC-myAnswers.txt}} file in \textbf{Lab 6} repository.
	
	\subsection{Describe the difference between the Bcc and BLcc instructions, and why the difference is important.}
	BLcc is used to call subroutines, it saves the return address in R14 (the Link Register) before branching (R14 $\leftarrow$ R15).
	
	\subsection{Briefly describe the programming conventions associated with subroutines that are used in this course.}
	R0 – R3 will be used to pass parameter arguments during invocation. R0 will be used for the first (leftmost) parameter, R1 for the second parameter, etc. Using only 4 (max.) registers to pass arguments limits the number of parameters that a subroutine can have. This course will not be concerned with how to pass more than 4 arguments (it would involve using some memory to store the extra arguments). R0 will also be used to return the return value.

\end{document}